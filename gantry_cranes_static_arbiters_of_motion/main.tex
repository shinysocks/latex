\documentclass[12pt]{article}

% Margin - 1 inch on all sides
\usepackage[letterpaper]{geometry}
\geometry{top=1.1in, bottom=1.1in, left=1.1in, right=1.1in}

\usepackage{fontspec}
\usepackage{tgtermes}
\usepackage{hanging}

\setmainfont[
 ItalicFont={texgyretermes-italic.otf},
 ]{texgyretermes-regular.otf}


% Doublespacing
\usepackage{setspace}
\doublespacing

% Outline stuffs
\usepackage{outlines}
\usepackage{enumitem}

\setenumerate[1]{label=\Roman*.}
\setenumerate[2]{label=\Alph*.}
\setenumerate[3]{label=\roman*.}
\setenumerate[4]{label=\alph*.}

% Begin document
\begin{document}

% First page name, class, etc
\noindent
Noah Dinan \\ Mark Zimmermann \\ COM1001 \\ \today \\

% Title
\begin{center}
Gantry Cranes: Static Arbiters of Motion
\end{center}

% Indentation
\setlength{\parindent}{0.5in}

While gantry cranes may seem like an obscure topic for an essay, I think these machines carry significant intrigue.
Gantry cranes are large, primarily stationary, cranes which are depicted carrying a semi trailer and in a shipyard
in paintings \textit{"Piggybacker" Loading Milwaukee Road Flatcar} by Albert W. Miller and \textit{Norwegian Dawn, Meyer-Werft Ship Yard} by 
Hans Dieter Tylle respectively. The prominence of gantry cranes in the transportation industry is, in a way, ironic because of their static nature.
These cranes often enable movement despite mostly staying still.

The first painting that caught my eye in the gallery is actually on the way to my college writing class. The bright yellows of the 
"Piggybacker" crane stood out to me as a clear choice for a painting to study further. \textit{"Piggybacker" Loading Milwaukee Road Flatcar}
was made by Albert W. Miller in 1968. This oil painting depicts a gantry crane, nicknamed "piggybacker" loading a semi truck trailer onto
a flatbed train car. The crane is a bright emergency yellow with green and blue highlights. The bright piece has more yellows reflected off the ceiling
and ground with some distant reds of a trainyard. The semi trailer being loaded contains the text "MILWAUKEE" which makes the location
of the painting clear. The style of this oil painting feels like impressionism with its visible brush strokes and more abstract coloration with shadows painted 
on in bright greens and blues instead of more tradional dark hues.

The second painting chosen is called \textit{Norwegian Dawn, Meyer-Werft Ship Yard} and was created by Hans Dieter Tylle, a German painter, in 1954.
I chose this painting while looking for other paintings that could connect to \textit{"Piggybacker"}. The massive green gantry crane which spans the enormous
ship central to this painting was a detail I couldn't miss. Nestled in a corner of the second floor of the Grohmann Museum, I was elated when I discovered
such clear connections between the two paintings. 
Besides the gantry crane, the style of this painting is also quite similar to Miller's \textit{"Piggybacker"}, both employing a 
near impressionistic oil technique with bright primary colors. Contrasting with Miller's work, this painting contains primarily blue colors, especially in
the scaffolding laced above the ship. This shipyard is likely located in Germany with Meyer-Werft being a German shipbuilding company.

A gantry crane is a special type of overhead crane which can vary greatly in size but typically has a static base and moving winch mechanism.
Gantry cranes are used in many manufacturing and transporatation industries for hoisting building materials, large vehicles, or other cumbersome
objects; even for facilitating art installations. Within the paintings chosen, a gantry crane is shown lifting a semi trailer onto a train car
in a railyard setting as well as a much larger crane likely unloading cargo from the massive ship in Tylle's \textit{Norwegian Dawn}. These two different
uses of the crane present an interesting similarity. Both cranes are encouraging some form of transportation, be it a trailer or shipping containers. Without
transportation, it is impossible for diversity to exist in industry. Transportation allows materials to be moved from all around and enables manufacturing in
locations where it may not have previously been possible. The usage of ships on the Great Lakes, for example, allowed industry to flourish in Milwaukee which
reveals a major connection between the freighter depicted in Tylle's painting and the "MILWAUKEE" semi trailer shown in Miller's piece. From massive ships, to
railyards to semi trailers, gantry cranes are significant for encouraging each step of the transportation process. 

One form of transportation which is mostly unused in industrial settings but remains my favorite pastime, is cycling. A bike shop I've been going to for
years to get flats changed and see the latest road bikes is Wheel and Sprocket Bike Shop in Bay View. Aside from supportive staff and neat bikes to look at,
built into the roof of this shop is a large yellow
gantry crane similar to the "piggybacker" depicted in Miller's work. This is because a building that now houses hundreds of bikes was formerly a lumber equipment manufacturing plant. The Wisconsin-based company that owned the building was called Cream City Iron Works and 
manufactured various iron parts for the lumber industry. Although the crane no longer hoists newly produced engines and saw blades, its legacy of empowering
movement lives on through the various bikes sold and repaired from beneath its iron gaze. 

While gantry cranes and other types of large machinery are ancient in concept, they are still in use and even commonplace in industrial spaces. Relevant to my major,
software engineering, automation is a topic gaining traction surrounding industrial machines. Automation allows large machinery to be controlled with little to no 
human interaction. This has various benefits towards safety, efficiency, and scale. In terms of future careers to look forward to, automating large machines such
as gantry cranes would certainly make the top of my list. Having control over such a powerful machine must truly make one feel akin to a god.

To conclude, within \textit{"Piggybacker" Loading Milwaukee Road Flatcar} by Albert W. Miller and \textit{Norwegian Dawn, Meyer-Werft Ship Yard} by 
Hans Dieter Tylle a special type of crane known as a gantry crane, is present. These cranes caught my eye because I had seen an uncannily similar machine housed
in the rafters of my favorite bike shop, Wheel and Sprocket. Within the two paintings, different cranes are shown with some similarities. Both cranes are
assisting with some form of transportation, despite being stationary themselves. This irony stood out to me as significant and powerful and remained true
in the Wheel and Sprocket bike shop as well. In my own future, I could foresee myself programming software to automate and control such machines, enabling
massive cranes to continue facilitatating movement.

% Bibliography
\newpage

\begin{center}
Works Cited
\end{center}

\begin{hangparas}{0.5in}{1}

Tanzilo, Bobby. “Wheel \& Spocket Spins a New Style of Bike Shop in Bay View.”
OnMilwaukee, 8 Nov. 2019, https://onmilwaukee.com/articles/wheel-and-sprocket-bay-view. Accessed 05 Nov. 2024.

“Gantry Crane: What Is It? How Is It Used? Types, Classes.” Gantry Crane, Industrial Quick Search,
https://www.iqsdirectory.com/articles/crane/gantry-cranes.html. Accessed 05 Nov. 2024. 

“Meyer Werft: Seven Generations in Papenburg.” Meyer Werft,\\
    https://www.meyerwerft.de/en/company/we\_are\_the\_meyer\_werft/index.jsp. Accessed 05 Nov. 2024. 

\end{hangparas}
\end{document}
