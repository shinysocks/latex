\documentclass[12pt]{article}

% Margin - 1 inch on all sides
\usepackage[letterpaper]{geometry}
\geometry{top=1.0in, bottom=1.0in, left=1.0in, right=1.0in}

\usepackage{fontspec}
\usepackage{tgtermes}
\usepackage{hanging}

\setmainfont[
 ItalicFont={texgyretermes-italic.otf},
 BoldFont={texgyretermes-bold.otf},
 ]{texgyretermes-regular.otf}

% Doublespacing
\usepackage{setspace}
\doublespacing

% Table
\usepackage{tabularx}

% Begin document
\begin{document}
\pagenumbering{gobble}

% Title
\begin{center}
\topskip0pt
\vspace*{\fill}
Lab \#12\\
November 22nd, 2024\\
Denaturation of Proteins and Enzymes\\
Noah Dinan, William Diodati, Nathan Cobb\\
\vspace*{\fill}
\end{center}

\newpage
\pagenumbering{arabic}

% Indentation
\setlength{\parindent}{0in}

\textbf{ABSTRACT}\\
For this Lab, our group wanted to examine some of the visual aspects of proteins being denatured. 
Enzymes are a distinct type of protein which typically catalyze reactions. They are the type
of protein we are denaturing for this lab. Four experiments were conducted by the lab groups to
see the results of enzymes denaturing as well as enzymes that experienced no reaction.
We saw enzyme denaturing due to pH change, the presence of organic compounds or heavy metal, and heat for enzymes in foods.\\

\textbf{INTRODUCTION}\\
The purpose of this experiment was to see some of the chemical changes that occur when an 
enzyme is denatured. The was accomplished through the process of four experiments. The pineapple and
gelatin experiment, where enzymes are denatured with heat, the pudding experiment where amylase in saliva denatures
enzymes in pudding cups, the milk and lemon juice experiment, where lemon juice changes the acidity of milk
and heavy cream, and the egg albumin experiement, which denatures proteins using, heavy metal, heat, acid, base, and alcohol.

Denaturing of proteins is a process by which proteins lose their secondary, tertiary, or quaternary structure resulting
in different chemical properties. While some of the experiments used controls, for most, a chemical change can be expected from
denaturing various foods that contain enzymes.\\

\textbf{MATERIALS AND METHODS}\\
The two experiments that our group conducted were the pudding saliva, and egg albumin one.
For these experiments we used:

packaged puddings (2)\\
spoons (2)\\
spreading map\\
A hot plate\\
6 large test tubes with egg albumin\\
10\% HNO3\\
10\% NaOH\\
95\% ethyl alcohol solution\\
10\% AgNO3\\

\textbf{EXPERIMENTAL PROCEDURE}\\
For the pudding and saliva experiment, the procedure was fairly straightforward.\\
Step 1: Separate the contents of a pudding cup into two beakers evenly.\\
Step 2: Put some pudding from one beaker into your mouth before spitting it back into the beaker.\\
Step 3: Stir both cups for 2 minutes\\
Step 4: Examine results, specifically observe the difference in thickness between the puddings.\\
\\
For the egg albumin experiment:\\
Step 1: Place 2 mL of egg albumin solution into each of the 6 large test tubes.\\
Step 2: Place one test tube in a hot water bath\\
Step 3: Add 2 mL of 10\% HNO3 to one test tube.\\
Step 4: Add 2 mL of 10\% NaOH.\\ 
Step 5: Add 4 mL of a 95\% ethyl alcohol solution.\\
Step 6: Add 2 mL 1\% AgNO3.\\ 
Step 7: Examine and compare results for different methods of denaturing.\\

\newpage

\textbf{RESULTS}\\
Observations for part A:\\
\begin{tabularx}{1\textwidth} { 
  | >{\centering\arraybackslash}X 
  | >{\centering\arraybackslash}X | }
 \hline
      & \textbf{Observations} \\
 \hline
    \textbf{Fresh Pineapple} & Darker yellow, gelatinous (should be more liquidly, 
    would be if it was fresh, enzyme in pineapple that would break 
    down the protein isn’t nearly as strong due to not being fresh)  \\
\hline
    \textbf{Canned Pineapple} & Light yellow gelatin, maintains gelatinous texture \\
\hline
    \textbf{Control} & Clear gelatin, keeps together well \\
\hline
\end{tabularx}\\

Observations for part B:\\
\begin{tabularx}{1\textwidth} { 
  | >{\centering\arraybackslash}X 
  | >{\centering\arraybackslash}X | }
 \hline
      & \textbf{Observations} \\
 \hline
    \textbf{Pudding without spit} & The pudding kept together in a glob when put onto the spreading mat \\
\hline
    \textbf{Pudding with spit} & The pudding dispersed gradually, forming a small puddle similar to how water would. \\
\hline
\end{tabularx}\\

\newpage

Observations for part C:\\
\begin{tabularx}{1\textwidth} { 
  | >{\centering\arraybackslash}X 
  | >{\centering\arraybackslash}X
  | >{\centering\arraybackslash}X | }
 \hline
      & \textbf{Observations} & \textbf{PH} \\
 \hline
    \textbf{Milk} & Looks like milk & 6 \\
\hline
    \textbf{Heavy Cream} & Similar to milk, some very small bubbles & 5 \\
\hline
    \textbf{Lemon Juice} & Slightly transparent and yellow & 3 \\
\hline
    \textbf{Milk with Lemon Juice} & Looks yellow and chunky, very thick & 3 \\
\hline
    \textbf{Heavy Cream with Lemon Juice} & Some curding on the surface, very thick and slightly yellow but opaquer than the milk with lemon & 3 \\
\hline
\end{tabularx}\\

Observations for part D:\\
\begin{tabularx}{1\textwidth} { 
  | >{\centering\arraybackslash}X 
  | >{\centering\arraybackslash}X | }
 \hline
      & \textbf{Observations} \\
 \hline
    \textbf{Heat} & Hardening and becoming a similar substance to cooked egg white \\
\hline
    \textbf{HNO3 (acid)} & Much cloudier than control \\
\hline
    \textbf{NaOH (base)} & Completely clear \\
\hline
    \textbf{Ethanol} & A bit cloudier, more dense \\
\hline
    \textbf{AgNO3 (Heavy Metal)} & Very opaque / cloudy \\
\hline
    \textbf{Control} & Maintained liquid consistency and mostly transparent color \\
\hline
\end{tabularx}\\

\newpage

\textbf{DISCUSSION}

\setlength{\parindent}{0.5in}
For part A, the gelatin with fresh pineapple was darker in coloration than the canned pineapple.
There would have been more significant differences in the structural integrity of the two pineapples if
we had had access to real fresh pieapple in lab. The lack of real fresh pineapple certainly skewed our 
results slightly for this experiment but we were still able to detect some denaturing of enzymes in pineapple 
from the breaking down of the gelatinous texture.
\par
For part B, the pudding without spit seemed to have an easier time staying together when placed upon the 
spreading mat. The pudding with spit seemed to behave more similarly to watery and spread out on the mat.
This is likely due to the amylase enzyme present in saliva breaking down the structures in the pudding.
Were fresh pineapple used in part A of the experiment, we would have seen similar results there as well, 
with the enzymes in the pineapple breaking down the gelatin and become more liquidy.
\par
In part C, the addition of lemon juice to milk and heavy cream caused them to become more thick and 
lowered the PH to be more similar to that of lemon juice (representing an increase in acidity).
We can see the PHs of milk and heavy cream 
reduce from 6 and 5 to 3. The thickness of both dairy products changes as well, with both becoming more 
thick and displaying signs of curding.

Finally, for part D, egg albumin, which is often used as a stabilizer similar to gelatin, is combined with
various reactants. Heat causes the albumin to harden, somewhat similar to the texture of a cooked egg.
adding HNO3 increased the cloudiness of the mixture, whereas NaOH turned it completely transparent.
Ethanol thickened the albumin and AgNO3 turned the mixture the most opaque. Most of the chemical changes
within this part of the experiment are revealed through changes in color.\\

\setlength{\parindent}{0in}
\textbf{CONCLUSION}\\
Within these experiments we witnessed chemical changes due to proteins being denatured within foods.
Results showed successful denaturing in the form of chemical changes for the different foods.
We can see enzymes breaking down the structure of gelatin and turning it into a more liquidy substance,
as well as coloration differences for egg albumin.

\end{document}
