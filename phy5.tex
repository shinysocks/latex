
   %%%%%%%%%%%%%%%%%%%%%%%
 %%%  NOAH'S SUPER COOL  %%%
%%%%      ACADEMIC       %%%%
 %%%   LATEX TEMPLATE    %%%
   %%%%%%%%%%%%%%%%%%%%%%%

\documentclass[12pt]{article}
\usepackage[letterpaper]{geometry}
\geometry{top=1in, bottom=1in, left=1in, right=1in}
\usepackage{fontspec}
\usepackage{tgtermes}
\usepackage{hanging}
\setmainfont[
 ItalicFont={texgyretermes-italic.otf},
 BoldFont={texgyretermes-bold.otf},
 ]{texgyretermes-regular.otf}
\usepackage{setspace}
\doublespacing
\usepackage{outlines}
\usepackage{enumitem}
\setenumerate[1]{label=\Roman*.}
\setenumerate[2]{label=\Alph*.}
\setenumerate[3]{label=\roman*.}
\setenumerate[4]{label=\alph*.}
\begin{document}

% Title Page
\pagenumbering{gobble} % remove page numbers
\begin{center}
\topskip0pt
\vspace*{\fill}
Pendulum Lab 5 \\ Noah Dinan \\ PHY 1110 - Mayer \\ \today \\
\vspace*{\fill}
\end{center}

\newpage
\pagenumbering{arabic} % resume page numbering

\setlength{\parindent}{0in}

\textbf{Introduction}

In this lab, we used equations of pendulum motion in order to estimate a value for \textit{g}.
Because of the small angle used with the simple pendulum, the simple equation below can be used to
relate the length \textit{L} of the string and period \textit{T} for a pendulum.

\[T=2\pi \sqrt{\frac{L}{g}} \]

To find a value for \textit{g}, this equation can be reorganized and set equal to \textit{g}.

\[g = 4\pi^2 \frac{L}{T^2} \]

Based on our experimentation and trials of measurement with the pendulum,
we can estimate a value for \textit{g} and compare it to the accepted value.
In addition to finding \textit{g}, we also found the uncertainty in our value for \textit{g}
with the following equation.

\[ \frac{u_{g}}{g} = \sqrt{ [\frac {u_{L}}{L}]^2 + [2 \frac {u_{T}}{T}]^2} \]

\textbf{Experimental Technique}

We only had to perform minimal setup as the pendulum was already setup when we arrived in the lab.
Our process for performing the experiment was to first measure the string for the pendulum, trading
meter sticks to counteract systematic error. Next we marked 2 degrees on the wall with a piece of tape
to ensure that we were releasing the pendulum from the correct height. We released the pendulum and recorded the
time it took to complete 20 oscillations. To measure times, we used a stopwatch and rotated which group member 
performed the measurements.

We repeated this 5 times and then plugged our numbers into the equations
shown in the introduction to determine a value for \textit{g} with uncertainty.

\newpage
\textbf{Results}\newline
After measuring the length of the pendulum with several different meter sticks, we measured the following
in units of centimeters.

\[ 113.0, 112.6, 112.0, 112.4, 112.0 \]

We then averaged these values to get a value for L

\[ L = 112.4 cm \]

Afterwards, we began our trials, timing 20 pendulum oscillations. For five trials we measured
the following values in seconds

\[ 42.01, 42.39, 41.23, 41.80, 42.10 \]

We averaged these values and divided by 20 to find the time for one oscillation in seconds

\[ T = 2.095 \]

Values for \textit{L} and \textit{T} can then be inserted into the equation for \textit{g}
to find our experimental value for \textit{g}.

\[ g = 4\pi^2 \frac{112.4}{2.095^2} = 1011.02 \frac{cm}{s^2} \]

Values for uncertainty can also be plugged in to find the uncertainty in \textbf{g}.

\[ \frac{u_{g}}{g} = \sqrt{ [\frac {0.424}{112.4}]^2 + [2 \frac {0.022}{2.095}]^2} \]

\newpage
which results in

\[ u_{g} = 21.573 \]

\textbf{Conclusions}

To conclude, our final value for \textit{g} is

\[ g = 1011.02 +- 21.573 cm^-2 \]

which is fairly close to the accepted value for \textit{g} of $980 \frac{cm}{s^2}$.

Most systematic error was eliminated by using different meter sticks and rotating
who recorded times for the oscillations. 

\end{document}
