
   %%%%%%%%%%%%%%%%%%%%%%%
 %%%  NOAH'S SUPER COOL  %%%
%%%%      ACADEMIC       %%%%
 %%%   LATEX TEMPLATE    %%%
   %%%%%%%%%%%%%%%%%%%%%%%

\documentclass[12pt]{article}
\usepackage[letterpaper]{geometry}
\geometry{top=1in, bottom=1in, left=1in, right=1in}
\usepackage{amsmath}
\usepackage{fontspec}
\usepackage{tgtermes}
\usepackage{hanging}
\usepackage{gensymb}
\setmainfont[
 ItalicFont={texgyretermes-italic.otf},
 BoldFont={texgyretermes-bold.otf},
 ]{texgyretermes-regular.otf}
\usepackage{setspace}
\doublespacing
\usepackage{graphicx}
\graphicspath{ {./} }

\begin{document}

% Title Page
\pagenumbering{gobble} % remove page numbers
\begin{center}
\topskip0pt
\vspace*{\fill}
The Ballistic Pendulum Lab 10 \\ Noah Dinan \\ PHY 1110 - Mayer \\ \today \\
\vspace*{\fill}
\end{center}

\newpage
\pagenumbering{arabic} % resume page numbering
\setlength{\parindent}{0in}

\textbf{Results}

In this lab report, we utilized the theories of conservation of momentum and energy to calculate the velocity
of a projectile ($v_A$) using two different methods and compared the two values found.

To examine potential sources of error, as we did not use a level, the angle of the projectile launcher
may not have been perfectly level. We also may have had error from
improper measurement of height and distance for the second part of the experiment.

\vspace{0.5cm}

The first method for finding $v_A$ is by firing the steel ball into a pendulum which will result in an inelastic
collision. The steel ball and pendulum will obtain kinetic energy: $\frac{1}{2}I\omega^2$ which can then be used to
determine the initial velocity of the steel ball ($v_A$).

First, we measured the following constant values:
\begin{enumerate}
    \item the mass of the steel ball, $m_A$ = 0.0667 kg
    \item the mass of the pendulum, $m_B$ = 0.2093 kg
    \item the length of the pendulum, $L$ = 0.31 m
\end{enumerate}

After several trials firing the projectile into the pendulum, we measured an average value for $\theta$
to be $\text{35}\degree$. Using this value we can determine $h = L - L\cos(35) = 0.59 m$.

Based on the equation for kinetic energy, we can find $v\prime$ using the equation below
\[ v\prime = L\sqrt{\frac{2mgh}{I}} \]

Where $I = (m_A + m_B)L^2$, and $h$ is found above.

And then...
\[ v_A = \frac{(m_A + m_B)v\prime}{m_A} \]
\[ v_A = 4.49 ms^{-1} \]

\newpage
We can compare this to the value obtained by examining the projectile fired off of the tabletop into the catch box.
We found the horizontal distance from the launcher to landing point of the projectile was 2.33 meters and the vertical
distance was 1.21 meters.

Using these values we can find model $v_A$ using the equation below. 

\[ v_A = \frac{2.33}{\sqrt{\frac{1.21(2)}{g}}} \]
\[ v_A = 4.69 ms^{-1} \]

\textbf{Conclusions}

To conclude, our two values were very close whether firing the projectile into a pendulum or off a table.

\[ 4.49 ms^{-1} \text{ vs } 4.69 ms^{-1} \]

This follows the theories of conservation of momentum and energy as a consistent velocity means these two quantities were conserved
for the experiment.

\end{document}
