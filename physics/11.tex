
   %%%%%%%%%%%%%%%%%%%%%%%
 %%%  NOAH'S SUPER COOL  %%%
%%%%      ACADEMIC       %%%%
 %%%   LATEX TEMPLATE    %%%
   %%%%%%%%%%%%%%%%%%%%%%%

\documentclass[12pt]{article}
\usepackage[letterpaper]{geometry}
\geometry{top=1in, bottom=1in, left=1in, right=1in}
\usepackage{amsmath}
\usepackage{fontspec}
\usepackage{tgtermes}
\usepackage{hanging}
\usepackage{gensymb}
\setmainfont[
 ItalicFont={texgyretermes-italic.otf},
 BoldFont={texgyretermes-bold.otf},
 ]{texgyretermes-regular.otf}
\usepackage{setspace}
\doublespacing
\usepackage{graphicx}
\graphicspath{ {./} }

\begin{document}

% Title Page
\pagenumbering{gobble} % remove page numbers
\begin{center}
\topskip0pt
\vspace*{\fill}
Hidden Cylinder P \\ Lab 11 - Noah Dinan \\ PHY 1110 - Mayer \\ \today \\
\vspace*{\fill}
\end{center}

\newpage
\pagenumbering{arabic} % resume page numbering
\setlength{\parindent}{0in}

\textbf{Introduction}\vspace{1em}

For this lab, we were given two wooden blocks of differing dimensions and tasked with finding the
diameter of cylindrical cavity embedded within one of them. The blocks we recieved were labeled $P$.
We had access a digital scale to find mass, as well as a ruler and caliper for length measurements.

\vspace{1em}

Because we knew both blocks are made of the same material, we can find the density for both blocks
and whichever is more dense does not contain the cylindrical cavity. Using the found density,
we can solve for the missing volume and diameter for the cylinder.

\vspace{1em}

The most basic equation used in this experiment is the equation for density, shown below.

\[ \rho = \frac{ \text{mass} } { \text{volume} } \]
\vspace{0.5em}

From this equation, we can derive an equation better suited for solving for missing volume.
For the following equation, a subscript of $\text{\_}_1$ relates to the unmodified block and
a $\text{\_}_2$ relates to the block containing a cylindrical cavity.

\[ \rho_1 = \frac{m_2}{V_2 - \pi r^{2} (\frac{h_2}{2})} \] 
\vspace{0.5em}

We can rearrange this equation to solve for the radius of the hidden cavity and finally the 
diameter.

\[ r = \sqrt{ \frac{m_2 - \rho_1 V_2}{\pi ( \frac{h_2}{2} ) \rho_1} } \]
\vspace{0.5em}

Where $V_2 = (l_2) (w_2) (h_2)$.

\newpage

Throughout our calculations, we ensured that uncertainty was propagated using the
respective equations for various operations.

\vspace{1em}

For addition such that $z = x + y$
\[ u_z = \sqrt{ u_x^{2} + u_y^{2} } \]

For multiplication such that $z = xy$
\[ \frac{u_z}{z} = \sqrt{ (\frac{u_x}{x})^{2} + (\frac{u_y}{y})^{2} } \]

For squaring such that $z = x^{2}$
\[ \frac{u_z}{z} = \sqrt{ 2(\frac{u_x}{x}) } \]

\vspace{1em}

\textbf{Experimental Technique}\vspace{1em}

The first step for our procedure was determining the lengths, widths, heights, and masses of the
two wooden blocks. The length measurements were initially performed with a caliper but we were unable to
properly tare the instrument and switched to a ruler instead.
 
Our measurements are listed below in the \textbf{Results} section.

\vspace{1em}

After taking measurements we simply input the numbers into the equations derived in the
introduction to first find the value for density $\rho_1$ and then find the value for $r$.
This procedure is further outlined below.

\newpage

\textbf{Results}\vspace{1em}

Our measurements for this experiment:
\begin{itemize}
    \item the mass of the first block, $m_1 = 93.1$ grams
    \item the mass of the second block, $m_2 = 55.4$ grams
    \item the length of the first block, $l_1 = 8.1$ centimeters
    \item the length of the second block, $l_2 = 6.2$ centimeters
    \item the width of the first block, $w_1 = 6.3$ centimeters
    \item the width of the second block, $w_2 = 6.4$ centimeters
    \item the height of the first block, $h_1 = 2.1$ centimeters
    \item the height of the second block, $h_2 = 2.2$ centimeters
\end{itemize}

\vspace{1em}

The first step to finding our value for diameter was to find the
density of the first wooden block.

\[ \rho_1 = \frac{ m_1 } { (l_1) (w_1) (h_1) } \]

\[ 0.869 = \frac{ 93.1 } { (8.1) (6.3) (2.1) } \]

using this density, we can solve for $r$ and double it to get a value for diameter.

\[ r = \sqrt{ \frac{m_2 - \rho_1 V_2}{\pi ( \frac{h_2}{2} ) \rho_1} } \]

\[ r = \sqrt{ \frac{55.4 - (0.869) (107.2)}{\pi ( \frac{2.2}{2} ) (0.869)} } \]

\[ r = 2.38 \pm 0.401 \text{cm} \]

and we can easily convert to diameter:

\[ d = 2r = 4.76 \pm 0.401 \text{cm}\]

For this experiment we encountered possible experimental error while measuring
lengths for the blocks, especially considering we we're unable to use the
caliper. This was the only major source of error.

\vspace{1em}

\textbf{Conclusions}\vspace{1em}

To conclude, our result of $4.76 \pm 0.401 \text{cm}$ absolutely seems reasonable when considering
the dimensions of the wooden block and measurement tools. The wooden block set we were given was labeled $P$.

\end{document}
