\documentclass[12pt]{article}

% Margin - 1 inch on all sides
\usepackage[letterpaper]{geometry}
\geometry{top=1.1in, bottom=1.1in, left=1.1in, right=1.1in}

\usepackage{fontspec}
\usepackage{tgtermes}
\usepackage{hanging}

\setmainfont[
 ItalicFont={texgyretermes-italic.otf},
 ]{texgyretermes-regular.otf}


% Doublespacing
\usepackage{setspace}
\doublespacing

% Outline stuffs
\usepackage{outlines}
\usepackage{enumitem}

\setenumerate[1]{label=\Roman*.}
\setenumerate[2]{label=\Alph*.}
\setenumerate[3]{label=\roman*.}
\setenumerate[4]{label=\alph*.}

% Begin document
\begin{document}

% Title
\begin{center}
Outline
\end{center}

\begin{outline}[enumerate]
\1 Introduction
    \2 Thesis: Diverse perspectives are detrimental to the success of propaganda.

\1 Power of the television for propaganda
    \2 Vermes' Hitler character is upset that there is more than only one channel on the TV.
    \2 quote on pg. 63

\1 Sagan connection
    \2 "Spin more than one hypothesis. If there’s something to be explained, 
        think of all the different ways in which it could be explained" (Sagan, ?).

\1 Goebbels' Principles on Propaganda
    \2 "Propaganda must be planned and executed by only one authority."
    \2 connect to maybe a mention of Goebbels in the book

\1 Adichie connection
    \2 A single story (single perspective) provides power to propaganda
    \2 quote on hitler's single story perspective

\1 Connection to Orwell's novel 1984
    \2 How does The Ministry of Truth connect to propaganda and Goebbels'
       principles of propaganda

\1 Conclusion

\end{outline}

\newpage

% First page name, class, etc
\noindent
Noah Dinan \\ Mark Zimmermann \\ COM1001 \\ \today \\

% Title
\begin{center}
A Single Source of Truth
\end{center}

% Indentation
\setlength{\parindent}{0.5in}

Although excellently executed propaganda may give the illusion of absolute control, propaganda is
deeply dependent on certain conditions in order to thrive. In his novel, \textit{Look Who's Back}, Timur Vermes
writes about the potential outcomes of a fictional return of Adolf Hitler. Throughout the novel, Vermes'
Adolf Hitler character depicts several of the real Hitler's traits and villiany. One such trait is the 
deep respect of propaganda and especially how the presence of diverse perspectives are detrimental to the
success of propaganda.

As Vermes' Hitler character wakes up to find himself in the year 2011, some of his initial discoveries are
new modern forms of technology, especially the evolution of television. In one particular scene,
Vermes' Hitler character is upset by the number of channels available on the modern television, 
noting, "there [are] many more buttons on the little box besides the simple on/off one" (Vermes 63).
The Hitler character can understand the "wonderful, magnificent opportunity for propaganda" (Vermes 63) but
is disturbed that there are options presented to the user. This monologue clearly illuminates propaganda's 
reliance on a single stream of information to the public. If a citizen is given multiple sources of information,
multiple channels for example, they will have multiple perspectives to consider and no single source of truth
presented to them. While a single channel of information may be a propagandist's wet dream, it is a nightmare to
believers in critical thinking skills.

Excellent critical thinking, the enemy of propaganda, relies on many perspectives to succeed. Prolific writer, and
astronomer, Carl Sagan was a major advocate for scientific thinking. In his work,
\textit{The Demon-Haunted World: Science as a Candle in the Dark},
he enumerates several tools for "baloney detection" including the advice to
"Spin more than one hypothesis. If there’s something to be explained, 
think of all the different ways in which it could be explained" (Sagan [PAGE NUMBER]).
Sagan emphasizes the importance of coming up with multiple hypotheses, multiple ways of 
interpreting truth to be testing and eliminated one by one. This is the opposite of the principles of propaganda
which revolves around the idea that the only hypothesis spun by the source of information is to be regarded
as truth and no other hypotheses should even be considered. There are numerous occurences of this in Vermes'
novel with the Hitler character being obsessed with only his hypotheses for the world, and failing to consider any
other hypotheses. This is seen when the character interacts with an assistant pleasantly until learning
that her last name is Özlem, a common Turkish surname. Vermes' Hitler immediately hypothesizes her as a liar,
"I felt so decieved, so betrayed, that I wished I could leave this bogus woman behind" (Vermes 78).
Despite her having done nothing wrong, the Hitler character pegs her as this horrible liar because of the
racist stereotypes so deeply ingrained in his flawed hypotheses in his flawed pseudoscience.

One of the leader's in the development of propaganda, Joseph Goebbels was the minister of propaganda for the
Third Reich led by Adolf Hitler. Goebbels wrote several principles on propaganda which were discovered in a
diary which he kept. This diary was translated by Louis P. Lochner and analyzed by Leonard W. Doob, professor of 
psychology at Yale University. One principle which connects directly with the topic of this essay and is
reflected in Vermes' Hitler's beliefs is Goebbels'second principle that "Propaganda must be planned and
executed by only one authority." (Doob 423). Propaganda executed by a single authority ensures more
consistency than relying on multiple. As propaganda often relies on misinformation, consistency is crucial.
Vermes' Hitler character describes the types of films that would be aired on modern televisions if he had control of 
them. He describes nationalist propaganda obscured by romance which would result in, "run(ning) out of application forms
for the League of German Women" (Vermes 86). Here the character describes how media should be completely controlled,
even entertainment should be a loose facade for propaganda which compels the viewer to perform a certain action.
With a single authority controlling propaganda, all media becomes propaganda and free thinking becomes more and more
challenging. A single hypothesis spun by only one authority is bound to become like Chimamanda Adichie's single story.

In her TED talk, "The Danger of a Single Story", Chimamanda Adichie warns how developing a single story about
someone limits the ways you can see that person, and can severely impact our judgement of the true nature of someone. 
The example that she gives is of a family who worked to do housework for her family who her mother mentioned were poor.
Because she was young and easily impressionable, she wrote her "single story" of them being poor, and was shocked to 
discover a stunning work of art while visiting their home. She says, "[a]ll I had heard about them was how poor they were,
so that it had become impossible for me to see them as anything else but poor.
Their poverty was my single story of them" (Adichie). In Vermes' novel, the Hitler character is frequently suseptible to
his single stories of people which limit his perspectives. In his first interaction with some young boys after waking up,
he assumes they are Hitler Youth and nothing can change his mind, this is his single story of them. After getting nowhere
in conversation with the boys, the character remarks, "I expect my needs did not appear sufficiently pressing to the 
Hitler Youths" (Vermes 8). Despite their clear bewilderment at being considered Hitler Youths, Vermes' narrator is unphased
and upholds his single story of them. Connecting back to propaganda, a single story is exactly the goal that any propaganda
wishes to accomplish. Creating a single, indisputable hypothesis about anything eliminates the need for alternative perspectives.

In George Orwell's renowned novel, \textit{1984}, the main character Winston Smith
works at an organization called the Ministry of Truth or Minitrue. This strange government ministry is responsible for
literally modifying information to control what is considered truth. "Winston's job was to rectify the original
figures by making them agree with the later ones" (Orwell [PAGE NUMBER]). Winston's and his collegues' job is to
literally forge past written information so that it aligns with current information. This is necessary in order to
ensure that anything said by The Party, Orwell's government entity, is maintained as total truth and never conflicted.
This aligns exactly with Goebbels' second principle of propaganda as The Party controls all of the information that exists
and all information can be modified according to their whims. This is an absolute form of propaganda where real scientific
truth is all erased. Winston mentions how, "[d]ay by day and almost minute by minute the past was
brought up to date. In this way every prediction made by the Party could be shown by documentary evidence to have been correct"
(Orwell [PAGE NUMBER]). If ever, a hypothesis presented by the Party became somehow under scrutiny, all documented evidence
would back up the single hypothesis. The Ministry of Truth is a organization of propaganda production at a level Goebbels
could only dream of.

To close, in Timur Vermes' novel, \textit{Look Who's Back}, a fictional Adolf Hitler character reveals much about the nuance
of propaganda, specifically how diverse perspectives cannot exist in this medium. This single source of propaganda is
connected to Joseph Goebbels second principle of propaganda, Chimamanda Adichie's single stories,
one of Sagan's baloney detection tools, and George Orwell's Ministry of Truth from the novel \textit{1984}.
Never let Propaganda own you, Adopt many ways of Thinking.. now there's a slogan!

% Bibliography
\newpage

\begin{center}
Works Cited
\end{center}

\begin{hangparas}{0.5in}{1}

%DO WE NEED TO CITE IN CLASS RESOURCES?
%Speaker last name, First name. "Talk title." TED, Month Year, URL.

Doob, Leonard W. "Goebbels’ Principles of Propaganda." The Public Opinion Quarterly,
vol. 14, no. 3, 1950, pp. 419–42. JSTOR, http://www.jstor.org/stable/2745999. Accessed 10 Dec. 2024.

Orwell, George. Nineteen Eighty-Four. Penguin Classics, 2021.

\end{hangparas}
\end{document}
