\documentclass[12pt]{article}
\usepackage[letterpaper]{geometry}
\geometry{top=1in, bottom=1in, left=1in, right=1in}
\usepackage{fontspec}
\usepackage{tgtermes}
\usepackage{hanging}
\setmainfont[
 ItalicFont={texgyretermes-italic.otf},
 BoldFont={texgyretermes-bold.otf},
 ]{texgyretermes-regular.otf}
\usepackage{setspace}
\doublespacing
\usepackage{outlines}
\usepackage{enumitem}
\begin{document}

\pagenumbering{arabic} % resume page numbering
\noindent
Noah Dinan \\ Grimes \\ Food Molecules \\ \today \\

\begin{center}
Final Poster Project Reflection
\end{center}

\setlength{\parindent}{0.5in}

For the final project in the Food Molecules course, our group created a poster describing the process of caramelization and its interaction with
our senses when we consume it. Our poster was split into three major sections, describing interaction with caramel and the nose, describing aromatic chemicals
that give caramel its signature flavor and smell, and finally a description of the process of caramelization and the different types of caramel produced.

For the first section, there is a description of how aromatic compounds present in caramel interact with our nose and trigger our memories and nostalgia.
One significant piece of this process is the Olfactory Epithelium which is a special part of the nose which translates smells into neural impulses that
the brain can register. The patterns in neural signals that this triggers are what gives caramel a familiar smell and triggers our prior memories and 
experiences. For the second section of the project, Aromatic chemicals are described in detail as well as a description of why we smell them so strongly.
These chemicals have benzene groups or modified benzene groups which are very stable and more likely to evaporate and be registered by our noses.

For the third section of our poster, I describe the steps in the caramelization process and what signifies different types of caramel. The three major steps of
caramelization are: Isomerization of fragmented sugars, Thermal Decomposition and fragmentation, and additional fragmentation and polymerization which leads
to the formation of aromatic flavor compounds and color compounds. The types of caramelization vary based on temperature from light caramel at 170 degrees C, to
darker and even unusable black caramel at higher temperatures.

Academically, for this project, I was happy with my research of the types of caramelization and the table that I created. I liked the visual idea of the
arrows for displaying consecutive steps as well. I was a bit disappointed at the lack of depth that I got into for each of the steps because there was not enough
room on the poster. Overall, I was proud of the poster visually, and we received positive feedback on the coloration. One area that did not go as well was
research trying to figure out the relationship between aromatic compounds present and the color of caramel. This was more of a point of interest than
a necesity though. Flan (which is depicted in the poster) is one of my favorite desserts with caramel and I thought the topic was interesting in that respect.

I was satisfied with working as a group, I think my group was responsible and overall had fairly good time management. Our biggest struggle was combining
all the information which we had researched separately into a coherent poster. I think breaking the poster up into sections successfully accomplished this goal.

\end{document}
