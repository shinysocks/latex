\documentclass[12pt]{article}

% Margin - 1 inch on all sides
\usepackage[letterpaper]{geometry}
\geometry{top=1.0in, bottom=1.0in, left=1.0in, right=1.0in}

\usepackage{fontspec}
\usepackage{tgtermes}

\setmainfont[
 ItalicFont={texgyretermes-italic.otf},
 ]{texgyretermes-regular.otf}


% Doublespacing
\usepackage{setspace}
\doublespacing

% Outline stuffs
\usepackage{outlines}
\usepackage{enumitem}

\setenumerate[1]{label=\Roman*.}
\setenumerate[2]{label=\Alph*.}
\setenumerate[3]{label=\roman*.}
\setenumerate[4]{label=\alph*.}

% Begin document
\begin{document}

% Title
\begin{center}
Outline
\end{center}

\begin{outline}[enumerate]
    \1 Introduction
        \2 Skepticism is critical to the strength of democracies and strong societies; authoritarian rule feeds on blind cooperation.

    \1 Body
        \2 What is skepticism and critical thinking? Why skepticism?
            \3 "The Fine Art of Baloney Detection"
                \4 Tools to form the basis of critical thinking
            \3 "Arguments from authority carry little weight–"authorities" have made mistakes in the past.
                They will do so again in the future" (Sagan 210)
                \4 Do not trust information merely due to its source.
            \3 "Propositions that are untestable, unfalsifiable are not worth much" (Sagan 211).
                \4 A hypothesis or proposition that is set up in a way to make it seem untestable does not signify that
                it is trustworthy or truthful.

    \1 Body
        \2 Democracies are strengthened by critical and scientific thinking
            \3 "We must be attentive and implement such public strategy to modernize our socialist country" (Sagan 18)
                \4 quote from chinese proclamation
                \4 masses can identity corruption in leaders 
            \3 "Encourage substantive debate" (Sagan 210)
                \4 improve hypotheses or generate new

    \1 Body
        \2 Authoritarian rule is strengthened by a lack of skepticism and critical thinking 
            \3 "Under the Tsars, religious superstition was encouraged, but scientific and skeptical thinking–except by a few
                tame scientists–was ruthlessly expunged" (Sagan 17).
                \4 Encouraging superstition makes a populace easier control and convince.
                \4 Skepticism and the ability to disprove lies and government manipulation is the enemy of dictatorships.
            \3 "uninformed cooperation (and often the cynical connivance)" (Sagan 13)
                \4 Gaining knowledge makes it harded to be ruled over and controlled.
                \4 becoming a "sucker" and the dangers of being a blind follower "sheeple" 

    \1 Body
        \2 Importance of critical thinking for societies and governmental strength
            \3 "become a nation of suckers" (Sagan 39)
            \3 "when governments and societies lose the capacity for critical thinking, the results can be catastropic" (Sagan 209)
                \4 national disconnect leading to less inteligent society: less prepared to tackle world problems.
            \3 "What kind of society could we create if, instead, we drummed into them science and a sense of hope" (Sagan 39)
                \4 importance of hope and a feeling of awe and wonder.

    \1 Conclusion

\end{outline}

\newpage

% First page name, class, etc
\noindent
Noah Dinan \\ Mark Zimmermann \\ COM1001 \\ \today \\

% Title
\begin{center}
Resisting Becoming a World of Suckers
\end{center}

% Indentation
\setlength{\parindent}{0.5in}

In an age where we are constantly being bombarded by information, it is important to have the sense to distinguish truth from fiction or deception.
Trust in authorities is important but should never be given blindly, without prior analysis. Societies lacking critical and scientific thinkers
are fraught with misinformation, distrust, and excessive governmental control. Skepticism is critical to the strength of democracies and strong societies; 
authoritarian rule feeds on blind cooperation.

Becoming a skilled skepticist and critical thinker is nuanced and does not require such extremes as
being a lie detector or appealing completely to distrust and pessimism. In his work, \textit{The Demon-Haunted World: Science as a Candle in the Dark}, Carl Sagan 
presents an admiration for scientific thinking and even devotes an entire chapter to "The Fine Art of Baloney Detection" (201). In this chapter,
Sagan enumerates his "baloney detection kit" (210), which presents tools of skepticism to be used "whenever new ideas are offered for consideration" (210).
I believe the ability to utilize this kit is an essential skill for participants in strong, resilient societies. Two specific tools mentioned 
carry relevance to skepticism for members of a society. Sagan emphasizes, "Arguments from authority carry little weight–'authorities' 
have made mistakes in the past. They will do so again in the future" (210). Trusting information blindly, basing its credibility on its source,
sets one up to be mislead and manipulated, especially by governments that thrive on total cooperation. The credibility of information should be based on 
how rigorously it has been tested, and if it is presented solidly and clearly. Experiments used to test an argument should be reproducible and clear, and not
even a little bit shady. If an argument is not easily tested and experimentation surrounding it can not be reproduced, 
it carries little worth, "Propositions that are untestable, unfalsifiable are not worth much" (Sagan 211).
A proposition that is set up as an inpenetrable truth, where attempting to disprove it is considered criminal, should not be considered valuable or trustworthy.
Governments seeking control often spew this sort of nonsense, be wary of untested, absolute truths.
Never trust information solely on who it's coming from; that which an emperor excretes is still shit.

Democratic societies are strengthened by critical and scientific thinking. For a form of government where deciding who will lead the country is left up to
the masses, backing decisions with skeptical thinking is necessary to ensure the society functions well.
Uninformed masses will inevitably elect uninformed leaders and lead a nation into turmoil. Sagan brings up a proclamation in which the government of China 
emphasizes the importance of scientific thinking, "We must be attentive and implement such public strategy
to modernize our socialist country" (18). Within this case, China's government realized how important scientific thinking is for a strong, modern country.
Separately from China's case, democracy ultimately relies on the strength of its members and especially their ability to identify 
corruption in leaders or organizations that may not have
the general public in mind. Skepticism is an excellent tool for identifying corruption as corruption is generally built on lies or an abscence of truth.
Unlike authoritarian rule, which thrives on masses that do not question, democracy encourages questioning and debating different ideas.
Within his tools for "baloney detection," Sagan also implores the reader to, "Encourage substantive debate" (Sagan 210).
Speculation and debate on arguments and hypotheses ultimately strengthens them and can even bring to light new ideas. This is of value to democracies
as they do not need to control their populace with absolute truths, as authoritarian governments do.

Contrasting with the values of democracies, authoritarian rule is strengthened by a lack of skeptical and critical thinking.
An example of this is shown in \textit{The Demon-Haunted World} when Sagan mentions a former
Russian empire's political tactics, "Under the Tsars, religious superstition was encouraged, but scientific and skeptical thinking–except by a few
tame scientists–was ruthlessly expunged" (17). By encouraging religious superstition and a belief in psuedoscience, governments have more
control of what information is widely accepted as truth. The reason for the Tsars expunging such a valuable skill informs about the nature
of dictatorships and similarly aligned governments. Skepticism and scientific thinking allows for the special ability to disprove misinformation
which is exactly what propaganda is. If the members of a society built on lies and government manipulation were given the tools
to discover this "baloney," they would certainly become unruly. Worse than a lack of critical thinking is a belief in psuedoscience.
Sagan warns of "uninformed cooperation" (13), such as following something that purports to be science, but is not. He warns of 
becoming a "sucker" (Sagan 39), one who has little say in how they're treated and lacks the knowledge to see past a wall of
misinformation, propaganda, and a total dearth of skepticism. To blindly accept is to truly become the mutant "sheeple", following the 
whims of the herd or a single "shepherd".

Besides the concerns of being controlled by a government or another single entity, living a life free of skepticism presents other
issues of societal, national, and even global strength when considered at length. To expand on Sagan's usage of the word, "sucker," he
explains, "if we don't practice these tough habits of thought, we cannot hope to solve the truly serious problems that face us–and we
risk becoming a nation of suckers, a world of suckers, up for grabs by the next charlatan who saunters along" (38-39). Become a nation or
world of suckers means relying on some leading entity as a source of knowledge, rather than our own thinking and judgement.
A world of suckers cannot innovate past our climate crisis, or resolve gender equality. In our current world, there are suckers present all around
us. Take a generation's total obsession with TikTok, a social media app built to be addicting and endlessly entertaining. Users of the app
engrossed in its splendors couldn't care less about the critical state of the world around them. 
The app commands total control of our attention and threatens to
transform an entire generation of prospective thinkers into mindless suckers. A world lacking critical thinkers is 
apocalyptic, "when governments and societies lose the capacity for critical thinking, the results can be catastropic" (Sagan 209).
This national and international disconnect leads to a less aware and actionable society, less prepared or willing to tackle
issues of global significance. For younger minds, Sagan argues the types of media they're exposed to value pseudoscience and
other malignant content over science and hope. He ponders, "[w]hat kind of society could we create if, instead, we drummed 
into them science and a sense of hope" (Sagan 39). Emphasizing science ultimately teaches the importance of hope and 
summons feelings of awe and wonder for children and adults alike.

To conclude, skepticism, critical thinking and scientific thinking tools are crucial to bringing strength to democracies and weakening authoritarian
control. Sagan's tools for "baloney detection" present skills necessary to disprove both incorrect information and misinformation.
A populace lacking these tools not only leads to stricter forms of government, but also weaker societies in general. A society filled with
those who benignly accept information is not the same society that can solve widespread problems and innovate past apocalypse.
To become a sucker is deadly; "[g]ullibility kills" (Sagan 218).

\end{document}
