
   %%%%%%%%%%%%%%%%%%%%%%%
 %%%  NOAH'S SUPER COOL  %%%
%%%%      ACADEMIC       %%%%
 %%%   LATEX TEMPLATE    %%%
   %%%%%%%%%%%%%%%%%%%%%%%

\documentclass[12pt]{article}
\usepackage[letterpaper]{geometry}
\geometry{top=1in, bottom=1in, left=1in, right=1in}
\usepackage{fontspec}
\usepackage{tgtermes}
\usepackage{hanging}
\setmainfont[
 ItalicFont={texgyretermes-italic.otf},
 BoldFont={texgyretermes-bold.otf},
 ]{texgyretermes-regular.otf}
\usepackage{setspace}
\doublespacing
\usepackage{outlines}
\usepackage{enumitem}
\setenumerate[1]{label=\Roman*.}
\setenumerate[2]{label=\Alph*.}
\setenumerate[3]{label=\roman*.}
\setenumerate[4]{label=\alph*.}
\begin{document}

\pagenumbering{gobble}

% Alternative snippet to title page
\newpage
\noindent
Noah Dinan \\ Yun Dong \\ UXD 1001 \\ \today \\

\begin{center}
Reflection I
\end{center}

\setlength{\parindent}{0.5in}

Between the three articles read, there are several similarities and ways in which articles complement
eachother. In the article titled, \textit{Intelligent Interfaces}, it is shown that the interaction between a human
and a machine must not always use traditional physical interactions but can instead use gestures and speech-based interaction
as well. While conversational or gesture-based interactions can be very useful, they also introduce concerns for the
privacy of users. As more data is collected in order to increase the ease of interactions, there is more risk
in the event of a data breach. On the flip side, for a traditionally "dumb" interface, there is less risk of an attacker gaining
access to personal data.

The article, \textit{Human Factors in Design}, discusses three major factors to consider when designing for humans.
The first human factor mentioned is that recognition beats recall. While interacting with a system, it is always better to
use recognition more than remembering how to interact with a system. When trying to perform a certain action with a system,
a well designed system will make the functions of buttons clear without needing to recall what each button performs.
When a user is presented with a list of information, they will always tend to remember best what is presented first and last.
This means that it is important to put the most important information early on in a list. The article also mentions affordances
which are tendencies that a user may have when interacting with a system. When designing an interface, it is crucial to use visual
and tactile clues for how a user should perform an interaction. These positive affordances greatly improve a product's usability.

The final article discusses Jakob's Law, an important concept often applied to the design of websites.
Jakob's Law roughly says that familiarity is extremely important to consider when implementing changes to a website interface
or designing a new website. When a user is familiar with the presentation of a website, they are more likely to be able to use it
without needing to recall how. As mentioned in an earlier article, this places recognition above recall.

The primary goal for all of these articles with  respect to UX, is to improve the efficiency of a user's interaction with a system.
Another important similarity, especially for interfaces which collect complex user data, is the risk of data breaches. When systems
collect too much user data, they can make excellent predictions but present Ethical Concerns because of
create massive privacy risks in the event of a breach of the information.

A specific example of a product I have recently used that fails to successfully use Human Factors and does not consider cognitive
factors are the light switches in Dierck's hall classrooms. These complex switches allow the user to different areas of lights and
the brightness of the lighting. The issue with the design of the switches is that there are not clear affordances that allow the
user to determine which buttons control which lights and it also is difficult to interpret which area of lighting is actually toggled on.

\end{document}
