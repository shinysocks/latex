
   %%%%%%%%%%%%%%%%%%%%%%%
 %%%  NOAH'S SUPER COOL  %%%
%%%%      ACADEMIC       %%%%
 %%%   LATEX TEMPLATE    %%%
   %%%%%%%%%%%%%%%%%%%%%%%

\documentclass[12pt]{article}
\usepackage[letterpaper]{geometry}
\geometry{top=1in, bottom=1in, left=1in, right=1in}
\usepackage{fontspec}
\usepackage{tgtermes}
\usepackage{hanging}
\setmainfont[
 ItalicFont={texgyretermes-italic.otf},
 BoldFont={texgyretermes-bold.otf},
 ]{texgyretermes-regular.otf}
\usepackage{setspace}
\doublespacing
\usepackage{outlines}
\usepackage{enumitem}
\setenumerate[1]{label=\Roman*.}
\setenumerate[2]{label=\Alph*.}
\setenumerate[3]{label=\roman*.}
\setenumerate[4]{label=\alph*.}
\begin{document}

\pagenumbering{gobble}

% Alternative snippet to title page
\newpage
\noindent
Noah Dinan \\ Yun Dong \\ UXD 1001 \\ \today \\

\begin{center}
Reflection 1
\end{center}

\setlength{\parindent}{0.5in}

    What are the main takeaways from each reading? [article]
     - Intelligent interfaces
        - Interacting with machines can be done with gestures and other traditionally human -> human interactions.
        - Invade privacy of consumers to make assumptions about emotional response
        - Conversational technologies / speech based interaction

     - Human Factors in Design [article]
        - Recognition beats recall (and working memory)
            + recognizing with icons rather than remembering how to do something is always easier
        - Primacy and recency effects
            + when listening from a list, people tend to remember things early on and at the end
            + place important information early on in a list that a user may read
        - Affordances are everywhere
            + how user thinks they should interact with a given system 
            + visual and tactile clues for how a user should perform an interaction

    - Jakob's Law [article]
        - Familiarity is most important when considering UX changes or
          when designing a new interface for a website.

    How do these readings complement or challenge each other?
     - improved efficiency (familiarity improves efficiency)
     - so much data presents increased risk for privacy related breaches
     - ethical risks to extended data collection for users

    Cognitive Bias in Design: The Human Factors in Design article introduces cognitive patterns that influence user behavior.
    Reflect on a product you’ve used that failed (or succeeded) because it did not consider these cognitive factors. What would you change about its design?
        - light button with in dierck's hall classrooms
        - it is unclear which light each button is associated with and if they are enabled or not
        - this is an example of a failed affordance

    Ethical Considerations in Design: AI-powered interfaces are becoming more predictive and autonomous.
    What ethical concerns arise when machines make decisions for users? How should designers balance personalization with user autonomy?
        - mass data = mass privacy risk in the event of a breach
    
    Select a product, interface, or service and analyze it based on insights from the readings.
    How could this product be improved using principles from Laws of UX or cognitive design factors?
    If you were designing an intelligent interface, how would you ensure it aligns with human needs rather than just technical capabilities?
        - iPhone Mini 12
        - This product could be improved by reintroducing bezels to the design and increasing overall durability
        - The phone is prone to being scratched when used without a case in a way that feels almost intentional to the design
        - another feature of a more tactile design could be physical toggles for user I/O such and camera or microphone toggles
            + this would increase the privacy
            + this would not impact the usability or familiarity of the product as much
              of those features are implemented through software rather than hardware.
\end{document}
