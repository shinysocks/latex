
   %%%%%%%%%%%%%%%%%%%%%%%
 %%%  NOAH'S SUPER COOL  %%%
%%%%      ACADEMIC       %%%%
 %%%   LATEX TEMPLATE    %%%
   %%%%%%%%%%%%%%%%%%%%%%%

\documentclass[12pt]{article}
\usepackage[letterpaper]{geometry}
\geometry{top=1in, bottom=1in, left=1in, right=1in}
\usepackage{fontspec}
\usepackage{tgtermes}
\usepackage{hanging}
\setmainfont[
 ItalicFont={texgyretermes-italic.otf},
 BoldFont={texgyretermes-bold.otf},
 ]{texgyretermes-regular.otf}
\usepackage{setspace}
\doublespacing
\usepackage{outlines}
\usepackage{enumitem}
\setenumerate[1]{label=\Roman*.}
\setenumerate[2]{label=\Alph*.}
\setenumerate[3]{label=\roman*.}
\setenumerate[4]{label=\alph*.}
\begin{document}

\pagenumbering{gobble}

% Alternative snippet to title page
\newpage
\noindent
Dong  ·  UXD 1001  ·  Noah Dinan

\begin{center}
    \textbf{Reflection II}
\end{center}

\setlength{\parindent}{0.5in}

From reading the articles and watching the Ted Talk, I gained insights on the widespread usage of storyboards.
It was interesting to learn how storyboards are used for drafting films and tv shows as well as in user experience design.
I also learned how important it is to consider all perspectives, not just your own.
In the case of the Ted Talk, Dietz didn't consider the perspective of a fearful child when designing an MRI machine.

The primary aspects of journey mapping and storyboarding that stood out to me were how
these tools can reveal a user's experience while interacting with a product. By creating a user
persona, a customer's theoretical interactions can be modelled and their emotions can be conveyed
to further understand a product's clients.

Storyboards and journey maps can be applied to develop better design ideas because they improve the
designers ability to empathize with end-users and create more human centered designs.
In addition, they can improve the accessibility of designs for different groups of people.

In the Ted Talk by Doug Dietz, he discusses building more human centered designs in medical settings.
Within my perspective, I hadn't considered the importance of comfort for such machines. Because so many
different types of people use medical equipment, it should be designed to be comfortable for humans.
In addition, especially for young children, the emotions of a user of the product must be considered to improve
interactions and user experience with a product.

If I was designing a product to be used for healthcare, I would primarily consider user comfort in designs.
For example, in chairs in a waiting room, I would make them comfortable for tired adults and soothing for
scared children. Improving the amount of cushioning and adding soothing colors would be some methods of
achieving this goal.

Designers may lack artistic talent and struggle to put their ideas into a storyboard or
may have difficulty creating accurate personas for their users. But ultimately these methods
provide an excellent framework for modeling user interactions and understanding your clients.

\end{document}
